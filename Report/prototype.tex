\label{prototype}
For the simulation model Netlogo was used as a simulation tool whereas the random data was generated with R.
In the following, the data model, the R script, the GUI and code from Netlogo will be described.
\subsection{Data model}
According to the problem description we defined a generic data model for every individual.
This data model is valid for the disco example, the university and labour market example, and contains the following attributes:
\begin{itemize}
	\item id: unique object identifier
	\item name: object name (e.g. male1 or female1)
	\item maxMatchesInt: integer value representing the maximum number of matchings for this object (e.g. in the disco example each person is maximally matched with one other person)
	\item sideInt: integer helper variable assigning an individual to one the two participant groups
	\item partnerList: list of identifiers of the partners, ordered according to the preference of an individual
	\item rankList: list of ranks (values between 0 and 1) for the partners from the partnerList 
	\item hasProposedToList: list of identifiers representing the individuals to which an individual has proposed
	\item gotProposedByList: list of identifiers representing the individuals from which an individual received proposals
	\item tmpMatchList: list containing the identifiers of individuals to whom a temporary match was established
	\item activeFlag: boolean value stating if an individual is matched or not respectively if the individual gave up (they already proposed to all individuals from the partnerList but only received rejections)
\end{itemize}

