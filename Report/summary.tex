\label{summary}
%\subsection{Subsection}
The algorithm by Gale and Shapley is very versatile and can be applied to different problems.
We have created a general model, which can apply this algorithm universally.
The only prerequisite is that the data is available and processed under the respective format.

In future work the algorithm can be applied to analyse the matching of students and universities or the labour market to pair supply and demand.
The major difficulty is data gathering and preprocessing. 
Statistics about university preferences are not available (for public use).
One possibility to overcome this problem is to use data about students who finished certain studies or to conduct surveys and make assumptions about their preferences and other factors (like the place of residence or previous education).

The code was written with runtime in mind, nevertheless the size of the data set had major impacts on the runtime.
Consequently in future work the performance of the simulation model has to be considered.

As mentioned above, it would be very interesting to test the model in further research projects with other data sets and areas of application.
Maybe, also extensions (like the optimal stopping) could be incorporated in future in more advanced versions of the simulation model.
