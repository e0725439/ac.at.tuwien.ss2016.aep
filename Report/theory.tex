\subsection{The Double Sided Matching Problem}
Clemens \& Flo will be here!

The double sided matching problem...

The stable matching problem refers to the problem of finding a matching between two sets of elements which may be equally sized. In \cite[p. 9]{gale62a} this problem is firstly described based on an example of college admission: a college is considering a set of $n$ applicants of which it can admin only a quota of $q$.

The assignment of students and colleges is not allowed to be unstable, i.e.\ there are two applicants $\alpha$ and $\beta$ who are assigned to colleges $A$ and $B$ although $\beta$ prefers $A$ to $B$ and $A$ prefers $\alpha$ to $\beta$. If this does not occur, the assignment is called \textit{stable}. In case there is more than one stable solution the \textit{optimal} one is of particular interest. In the previously mentioned college example a stable assignment is called \textit{optimal} if every applicant is at least well off as it would be under any other stable assignment \cite[p. 10]{gale62a}. In Economics this is also known as pareto efficiency \cite[p. 46]{9780199297818}.

\subsection{Gale and Shapley Algorithm}
Clemens \& Flo will be here!

\subsection{Relevance to Game Theory}
...work in progress...
\subsubsection{General}
"Game theory is about what happens when people - or genes, or nations - interact". An important aspect for participating parties is to anticipate how the opposite party will react on certain actions. Mathematics shall help to analyze, understand and estimate outcomes of such games. Depending on the information participants have, they choose how to act basing on rules contained in their strategy. Game theory is widely applied in economics. Companies use game theory to estimate e.g. reactions of competitors or behavior of employees. The major advantages of game theory are its precision and that it can be applied to analyze all kind of games. \cite[p. 1ff]{camerer2003behavioral}

In addition there are some assumptions which are made during preparation and execution by every individual who are interacting together \cite {gibbons1997gametheory}
\begin{itemize}
	\item Well-specified choices
	\item Well-defined end-state
	\item Specified payoff
	\item Perfect knowledge
	\item Rationality 
\end{itemize}

These perfect preconditions will never be met in real scenarios.

\subsubsection{Economic Applications}

\subsubsection{Double Sided Matching Problem and Game Theory}

Like described in chapter XXX, the three main problems which will be simulated in the course of this project, consist of two sides or parties which shall be brought together. In the original marriage problem discussed by Shapley the goal was to bring men and women together so the overall situation is stable (....). This idea can also be expanded to other areas like the labour market or university applications. In each of these three problems the parties need to have a strategy how to react to moves of the counterparty. If, for example, a university offers a place to a student and this is not the student's favorite university,  he/she has to decide whether to accept the place (to be on the safe side) or still hope also to get accepted from the favorite university. Not only the student needs to have a strategy how to deal with these situations but also universities have to incorporate them into their application process (e.g. how many students will be invited).