\label{theory}
In this chapter the necessary theoretical background for the realization of this project is discussed.

\subsection{The Double Sided Matching Problem}
The double sided matching problem describes a problem, which has two sets of elements.
These sets are disjoint and may be equal in size, but this is not mandatory.
The purpose of the double sided matching algorithm is to find a solution, so every element of the first set has a corresponding counterpart in the other set. This can be best described in an example:
Assume that the elements are men and women and that there are only heterosexuals preferences for the individuals.
The two sets are now male and female.
The double sided matching algorithm shall create couples, which consists of a man and woman.

The solutions of the matchings can have different properties, which might be very important (e.g. stability).
The output of the algorithm of men and women is not allowed to be unstable, i.e.\ there are two men $\alpha$ and $\beta$ who are assigned to women $A$ and $B$ although $\beta$ prefers $A$ to $B$ and $A$ prefers $\alpha$ to $\beta$.
If this does not occur, the assignment is called \textit{stable}.
If there is more than one stable solution the \textit{optimal} one can be of particular interest.\\ 
Optimality means that a solution has the best outcome for an individual or set.
In this case optimality can be achieved from the perspective of the first set of elements or from the second set.
There can also be a trade-off from the perspective of the both sets, it this case optimality is achieved by weighting the expected value of both sets.
These properties have been described in the paper by Gale and Shapley \cite{gale62a}.\\
Generally in economics this is also known as pareto efficiency \cite[p. 46]{9780199297818}. 
Pareto efficiency is a characteristic for a solution where any change which would improve one property on one hand, but will worsen another one on the other hand.


\subsection{Gale and Shapley Algorithm}

In the paper of Gale and Shapley \cite{gale62a} an algorithm was proposed for solving the double sided matching problem.
Lets assume there are two sets of elements: one contains men and the other one women.
The size of the two sets is equal, the preferences are heterosexual and there are no polygamies (i.e. one element is only allowed to be matched with one element from the other set).
The aim of the algorithm is to find a stable way of marrying men and women.
Each men and women has their own preference for their marriage partner.

\paragraph{First Iteration\\}
In the first iteration each man proposes to the woman who he ranked first.
Afterwards a woman has a list of men that proposed to her. This list might be zero, if not she rejects every men, except her highest preferred man on this list.
If a man is not rejected he and the woman create a temporary couple.
All man who are not in a couple are single.

\paragraph{Second Iteration\\}
In the second interaction each man who is single, proposes to his highest ranked woman, who he didn't already proposed to.
Each woman now chooses the man which she ranks the highest out of her received proposals.
If a man is not rejected he and the woman create a temporary couple (Note: after the second iteration some men, who have been in a couple from the first iteration might be now single).

\paragraph{Further Iterations\\}
The third iteration is the same as the second one.
This continues until no men is single any more.
If this condition is satisfied a stable solution is found for the double sided matching problem.
The stability of the solution of this algorithm has been proven by contradiction \cite{gale62a}.

\subsection{Relevance to Game Theory}

\subsubsection{General}
"Game theory is about what happens when people - or genes, or nations - interact" \cite[p. 1]{camerer2003behavioral}.
An important aspect for participating parties is to anticipate how the opposite party will react on certain actions.
Mathematics shall help to analyse, understand and estimate outcomes of such games. 
Depending on the information participants have, they decide how to act basing on rules contained in their strategy.
Game theory is widely applied in economics.
Companies use game theory to estimate e.g. reactions of competitors or behaviour of employees.
The major advantages of game theory are its precision and its universal applicability to all kind of games. \cite[pp. 1-3]{camerer2003behavioral}

In addition there are some assumptions which are made during preparation and execution by every individual who is interacting with another one:
\begin{itemize}
	\item Well-specified choices
	\item Well-defined end-state
	\item Specified pay-off
	\item Perfect knowledge
	\item Rationality 
\end{itemize}

However, these perfect preconditions will never be met in real scenarios. 
In the context of double sided matching, it is important to have well-specified choices defined. 
Nevertheless, there might be chances to get into a nearly worst-case scenario instead.
But furthermore, as a consequence of these listed assumptions above, there are two different kind of games:
 
\begin{itemize}
	\item Static ones
	\item Dynamic ones
\end{itemize}

The two items can be combined with the following kind of knowledge

\begin{itemize}
	\item Incomplete
	\item Complete (there is no private information)
\end{itemize}

Depending on the nature of the game, whether it is static or dynamic and the knowledge is complete or incomplete four different combinations can be made.
For a better understanding, following examples only consist of two players (every combination can be used for a higher amount of players too):

\begin{itemize}
	\item Static and (in)complete information\\
	As of complete information, the players action set can be presented in an N x M matrix.
	Therefore all information (also private) is known to every player and each player is able to easily eliminate the pay-offs of the 		
	other players. The pay-off is the outcome of a chosen action of a player which depends on the state or previous actions of 
	the game. The pay-off can change dynamically while the game is in progress.
	As of incomplete information players are forced to be uncertain about the other player's actions and pay-offs.
	Besides that, there are two basic steps
	\begin{enumerate}
		\item Player 1 and 2 are choosing actions out of a set, which were predefined by them before.
		\item After their actions, they receive their pay-offs.
	\end{enumerate}
	\item Dynamic and complete information\\
	When basic assumptions are made on complete information, players will create a strategy to ensure to get the highest return 		
	of each action. Although the strategy might be changed during the game after each turn.
	Incomplete information leads to communication of uninformed parties (i.e. closed private 
	information) which might end up to under- or overestimate the others pay-offs. In general there will be a Nash Equilibrium 
	between those parties. This equilibrium is created by players because they choose a strategy which do not depend on the other 	players. There are three basic steps: 
	\begin{enumerate}
		\item Player 1 chooses an action out of his/her predefined set.
		\item Player 2 observes player 1's action and chooses an action based on his/her observation.
		\item After their actions, they receive their pay-offs.
	\end{enumerate}
\end{itemize}

The previously described theory is based on information of the paper "An introduction to applicable game theory" \cite {gibbons1997gametheory}.

\subsubsection{Example for the Application of Game Theory}
Game theory is also very popular in the field of economics.
Many companies primarily apply game theory to estimate the reactions of their competitors. 
In the last auction for broadband internet frequencies in Germany, all major providers hired game theorists out of two reasons. 
Firstly, they wanted to win the auction for the desired frequency and secondly, they also did not want to overpay for the frequency.
The bidding behaviour of a provider indicated the interest rate for a frequency. 
As a consequence, other providers retained their bids for parts of this frequency so the interested provider did not need to overpay for it. 
The positive result for the providers was, that the 700-MHz frequency parts were sold for the minimum bidding price. \cite{gametheoryWelt}

\subsubsection{Double Sided Matching Problem and Game Theory}

Like described in chapter \ref{theory}, the goal of the three main problems, which will be simulated in the course of this project, is to bring different parties together.
In the original marriage problem discussed by Shapley, the goal was to match men and women so the overall situation is stable.
This idea can also be applied to other areas like the labour market or university applications.
In each of these three problems the parties need to have a strategy how to react to moves of the counterpart.
If, for example, a university offers a place to a student and this is not the student's favourite one, he/she has to decide whether to accept the place (to be on the safe side) or still hope to get accepted from the favourite university (and as a consequence accept to be on the waiting list there).
Not only the student needs to have a strategy how to deal with these situations but also universities have to incorporate different reactions of students into their application process (e.g. how many students to invite) \cite{gale62a}. \\
The double sided matching problem is a dynamic game.
The players of the two sets have incomplete information.
The algorithm uses the information of both sets, therefore it has complete information.

An important aspect of the game or matching is that the rules have to be clear, which was mentioned by Roth and Sotomayor.
The way agents are matched to each other influences the analysis of the problem.
A possible rule might be that individuals like a student and university are only brought together if both parties agree to the matching.
Other norms of a game might contain the way an individual proposes to another one or whether there exists a moderating individual \cite[p. 492]{roth1992two}.

\clearpage
\subsubsection{Double Sided Matching Problem and Optimal Stopping}
The double sided matching problem also relates to the optimal stopping problem. 
The optimal stopping problem concerns itself with choosing the optimal time to stop an undertaken action and at the same time to minimize the expected costs and maximize the profit. 
A concrete example for the optimal stopping problem could be a men trying to find a women to spend the rest of his life with. 
He can meet several different women but when is the optimal time to stop looking?

A few algorithms leading to a possible solution for the optimal stopping problem have been presented by Christian \& Griffiths 2016 \cite{algtoliveby}. 
For the previously mentioned example of partner search, one algorithm proposes to spend 37\% of the time in building standards. 
Afterwards, the man should immediately propose/choose the next woman he meets who satisfies or over-exceeds his standards.
37\% is the provably optimal solution \cite[p. 2]{algtoliveby}. 
This algorithm is recommended in the case of \textit{no-information games} (e.g. in the case of a partner search, it is hard to quantify the information, because it is hard to compare people to each other. There is no concrete criteria, one can only assume that one person is better or more suitable than another person, but not by how much) \cite[p. 18]{algtoliveby}.

Another possible solution for the previously mentioned problem is to set a threshold from the beginning and to take the first option that exceeds it. 
Christian \& Griffiths 2016 \cite{algtoliveby} recommended this approach in the case of \textit{full-information games}.
A suitable example for this case is the sale of a house: the value of the house can be specified clearly and there exists also information about the state of the market. 
The combined knowledge allows to predict a range of offers. 
In this case a threshold is set and the first offer that exceeds this threshold can be accepted.
This example does not include waiting costs. 
If waiting costs exist, they also have to be taken into account. 
Therefore, it is recommended to decline an offer if the chance of a better offer multiplied by how much better it is compensates the waiting costs.

Finally Christian \& Griffiths 2016 \cite[pp. 28-30]{algtoliveby} recommend to always stop looking at a certain point.



