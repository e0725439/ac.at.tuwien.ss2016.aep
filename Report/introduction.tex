This paper deals with the investigation of the double sided matching along with its application in economics. 
\todo{Add more introduction text}

\subsection{The problem}
In many daily live expierences there is an hidden double sided matching.
For example
\todo{Find a good example}
But we don't have the time or energy to find the perfect or optimal solution.
Whereas in some field this could be done more easily or because this is mandatory, like in the universities in germany.
\todo{Is this true?}
They have an admission restriction for the students.
Only the students with the best grandes in high school can go on the best universities in germany.
But these universitites can only take a limited amount of students.
\todo{Define the problem a bit better}


\subsection{The aim}
The aim of the paper is to develop a prototype, which could solve the previous discussed problem and similar ones.
With this software we could solve questions like:
What would be an optimal solution for the german students and universities?
\todo{Add more questions}

\subsection{Structure}
\todo{rework shortly before we are finished, because structure changes most time}
\todo{add dynamic linking to chapters}

The seconds chapter deals with the theory overview and background of the double sided matching problem.
We will take a closer look at the Gale and Shapley Algorithm and what this has to do with game theory.

The third chapter descripes our prototype, which implements the previous discussed theory into practice.

In the fourth chapter we use data from Austria with our prototype to make an statment over the college applications.

Chaper five deals deals with the labor market in the european union. Our prototype will get this data and we will discuss the results of it.

In the summary we will sum up the important findings and compare the outputs of our prototype.

