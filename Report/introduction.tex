\label{introduction}
Double sided matching is about finding a match between two sets of elements.
The first algorithm which should solve this problem was described by David Gale and Lloyd S. Shapley in 1962 \cite{gale62a}. 
The focus will lie on the examination of this algorithm and the creation of a simulation model for it, which can be applied to different economic issues.
The selected economic issues will be presented in the following chapters. 
\todo{add one sentence about something}

\subsection{Motivation and Problem Description}
In many daily live situations there are hidden double sided matching problems. 
For example students usually have the opportunity to apply to a number of schools. 
In this case students and schools have to be matched together. 
Both parties have preferences which need to be taken into consideration. 
In our everyday life we usually don't have the time to find an appropriate solution. 

In some cases these problems even go by without solving them. 
In other cases however, it is crucial to solve them, e.g. if we consider medical school graduates and open positions in hospitals. 
Due to multiple reasons (e.g. different preferences, high number of elements in the two sets) this problem cannot be solved intuitively or in linear time.

\subsection{Aim and Objectives}
The aim of this paper is to develop a simulation model prototype which can solve similar problems, like the ones discussed above.
With this model we could answer questions like:
\begin{itemize}
	\item Are there "stable" solutions for this kind of problem?
	\item How can it be applied to all prospective students and universities in Austria?
	\item Is there also a solution for the current labour supply and demand in the European Union?
\end{itemize}

\subsection{Structure}
\todo{rework shortly before we are finished, because structure changes most time}
In the \hyperref[theory]{second chapter} a theoretical overview is provided regarding the double sided matching problem and what stability means in this context.
The focus lies on the Gale and Shapley Algorithm, basic information about the game theory and the relationship between them.

The \hyperref[prototype]{third chapter} describes the prototype of the simulation model, which implements the previously discussed theory into practice. 
As a simulation environment NetLogo will be used.

The \hyperref[evaluation_students]{fourth chapter} deals with the examination of data from Austrian universities and their integration into the prototype. 
As a result, a solution for this double sided matching problem will be provided.

\hyperref[evaluation_labor]{Chapter five} deals with the labour market in the European Union, trying to match the labour demand and labour supply. 

Finally the most important findings and results are summarized and compared in the \hyperref[summary]{last chapter}.
Additionally there is a short outlook into further research.