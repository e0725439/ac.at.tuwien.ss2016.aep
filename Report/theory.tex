\label{theory}
\subsection{The Double Sided Matching Problem}
The double sided matching problem describes a problem, which has two sets of elements.
These sets are disjoint and may be equal in size, but this is not mandatory.
The job of the double sided matching algorithm is to find a solution, that every element of the first set has a corresponding counterpart in the other set. This can be best described in an example:
Assume that the elements are men and women and that there are only heterosexuals preferences for the elements.
The two sets are now male and female.
The double sided matching has now to create couples, which consists of a man and woman.

This solution of the matching can have some properties, which might be very important (e.g. stability).
The assignment of men and women is not allowed to be unstable, i.e.\ there are two men $\alpha$ and $\beta$ who are assigned to women $A$ and $B$ although $\beta$ prefers $A$ to $B$ and $A$ prefers $\alpha$ to $\beta$.
If this does not occur, the assignment is called \textit{stable}.
If there is more than one stable solution the \textit{optimal} one is of particular interest.
In this case optimality can be achieved from the perspective of the first set of elements or from the second set.
There can also be a trade-off from the perspective of the both sets, it this case optimality is achieved by weighting the expected value of both sets.
These properties have been described in the paper by Gale and Shapley \cite{gale62a}.
In Economics this is also known as pareto efficiency \cite[p. 46]{9780199297818}.


\subsection{Gale and Shapley Algorithm}

In the paper from Gale and Shapley \cite{gale62a} an algorithm was proposed for solving the double sided matching problem.
Lets assume there are two sets of elements: one with men and the other one with women.
The size of the two sets is equal and the preferences are heterosexual and that there is no polygamies (i.e. one element is only allowed to be matched with one element from the other set).
The aim of the algorithm is to find stable way of marrying the men and women.
Each men and women has their own preference for their marriage partner.

\paragraph{First Iteration\\}
In the first iteration each man proposes to the woman which he ranked first.
Afterwards each woman rejects all men which proposed to her, excepting her highest preferred man, which proposed to her.
If a man is not rejected he and his woman create a temporary couple.
All man which are not in a couple are single.

\paragraph{Second Iteration\\}
In the second interaction each man which is not in a couple, proposes to his highest ranked woman, which he didn't already proposed to.
Each woman now chooses the man which she ranks the highest out of her received proposals.
If a man is not rejected he and his woman create a temporary couple (Note: after the second iteration some men, which have been in a couple from the first iteration might be now single).

\paragraph{Further Iterations\\}
The third iteration is the same as the second one.
This continues until all men are in a couple.
If this condition is satisfied we have achieved a stable solution of the double sided matching problem.
The stability of the solution of this algorithm has been proven by contradiction \cite[p. 12]{gale62a}.

\subsection{Relevance to Game Theory}
...work in progress...

\subsubsection{General}
"Game theory is about what happens when people - or genes, or nations - interact".
An important aspect for participating parties is to anticipate how the opposite party will react on certain actions.
Mathematics shall help to analyze, understand and estimate outcomes of such games. Depending on the information participants have, they choose how to act basing on rules contained in their strategy.
Game theory is widely applied in economics.
Companies use game theory to estimate e.g. reactions of competitors or behavior of employees.
The major advantages of game theory are its precision and that it can be applied to analyze all kind of games. \cite[p. 1ff]{camerer2003behavioral}

In addition there are some assumptions which are made during preparation and execution by every individual who are interacting together \cite {gibbons1997gametheory}
\begin{itemize}
	\item Well-specified choices
	\item Well-defined end-state
	\item Specified payoff
	\item Perfect knowledge
	\item Rationality 
\end{itemize}

These perfect preconditions will never be met in real scenarios.
But furthermore, as a consequence of these listed assumptions above, there are two different kind of games:

\begin{itemize}
	\item Static ones and
	\item Dynamic ones
\end{itemize}

The two items can be combined with the following kind of knowledge

\begin{itemize}
	\item Incomplete
	\item Complete
\end{itemize}


\subsubsection{Economic Applications}
Game theory is also very popular in the field of economics.
Many companies primarily apply game theory to estimate the reactions of their competitors. 
In the last auction for broadband internet frequencies in Germany, all major providers hired game theorists to win the auction for the desired frequency but don't overpay it.
The bidding behavior of one provider indicated the interest rate for the frequency. 
As a consequence, other providers retained their bids for this frequency so the interested provider did not need to overpay for it. 
The result was, that the 700-MHz frequencies were sold for the minimum bidding price. \cite{gametheoryWelt}

\subsubsection{Double Sided Matching Problem and Game Theory}

Like described in chapter XXX, the goal of the three main problems, which will be simulated in the course of this project, is to bring two different parties together.
However, a requirement is that the matching of these parties is stable.
In the original marriage problem discussed by Shapley the goal was to match men and women so the overall situation is stable.
Stability in this case means that these men and women are in a better situation after the matching than they were before alone. 
This idea can also be expanded to other areas like the labour market or university applications.
In each of these three problems the parties need to have a strategy how to react to moves of the counterparty.
If, for example, a university offers a place to a student and this is not the student's favorite university, he/she has to decide whether to accept the place (to be on the safe side) or still hope to get accepted from the favorite university (and maybe accept to be on the waiting list there).
Not only the student needs to have a strategy how to deal with these situations but also universities have to incorporate different reactions of students into their application process (e.g. how many students will be invited) \cite {gale62a}.

An important aspect Roth and Sotomayor mention is, that the rules of the game/matching have to be clear.
The way agents are matched to each other influences the analysis of the problem.
A possible rule might be that individuals like a student and university are only brought together if both parties agree to the matching.
Other rules of a game might be the way one individual proposes to another, whether there exists a moderating individual and many more \cite[p. 492]{roth1992two}.