\label{prototype}
For the simulation model NetLogo was used as a simulation tool whereas the random data was generated with R.
In the following, the data model, the R script, the GUI and code from Netlogo will be described.
\subsection{Data model}
According to the problem description we defined a generic data model for every individual.
This data model is valid for the disco example, the university and labour market example, and contains the following attributes:
\begin{itemize}
	\item id: unique object identifier
	\item name: object name (e.g. male1 or female1)
	\item maxMatchesInt: integer value representing the maximum number of matchings for this object (e.g. in the disco example each person is maximally matched with one other person)
	\item sideInt: integer helper variable assigning an individual to one the two participant groups
	\item partnerList: list of identifiers of the partners, ordered according to the preference of an individual
	\item rankList: list of ranks (values between 0 and 1) for the partners from the partnerList 
	\item hasProposedToList: list of identifiers representing the individuals to which an individual has proposed
	\item gotProposedByList: list of identifiers representing the individuals from which an individual received proposals
	\item tmpMatchList: list containing the identifiers of individuals to whom a temporary match was established
	\item activeFlag: boolean value stating if an individual is matched or not respectively if the individual gave up (they already proposed to all individuals from the partnerList but only received rejections)
\end{itemize}

\subsection{Data generation}
In order to generate data for the disco example, the statistical computing language R\footnote{Available at \url{https://www.r-project.org/}, accessed April 28, 2016} has been used. 
This generates a CSV list, which is then read from NetLogo. 
The script accepts the following parameters:
\begin{itemize}
	\item seed: integer value, which is used as a seed by the number generator. The default value is 123.
	\item numberOfMen: integer value representing the number of men which should be created. The default value is 10.
	\item numberOfWomen: integer value representing the number of women which should be generated. The default value is 10.
	\item pickyLower: float value between 0 and 1 representing the lower bound which will be used in order to generate a number that represents how picky a person is. The value 0 means excepts every possible match, the value 1 excepts none.
	\item pickyUpper: float value between 0 and 1 representing the upper bound which will be used in order to generate a number that represents how picky a person is. The value 0 means excepts every possible match, the value 1 excepts none.
\end{itemize}

The script can be called as follows from the command line: 
\begin{verbatim}
Rscript initDisco.R 123 10 10 0 0
\end{verbatim}

The script generates a CSV which looks as follows:
\begin{verbatim}
"";"id";"name";"maxMatchesInt";"sideInt";"partnerList";"rankList"
"1";1;"male1";1;1;"13#18#14#17#16#11#20#19#12#15";"0.96#0.95#0.9#0.68#0.57#0.45#0.33#0.25#0.1#0.04"
...
"11";11;"female1";1;2;"3#9#5#4#10#8#2#1#6#7";"0.92#0.82#0.7#0.67#0.48#0.41#0.35#0.25#0.22#0.05"
\end{verbatim}


