\label{summary}
%\subsection{Subsection}
The algorithm by Gale and Shapley is very versatile and can be applied to different problems.
It would be interesting to also apply it to the labour market to pair supply and demand. 
Furthermore, it would also be interesting to see how good the simulation model performs with real life data (e.g. matching students and universities).

The major difficulty will probably be the data gathering and preprocessing. 
Statistics about university preferences are not available (for public use).
One possibility to overcome this problem would be to use data about students who finished certain studies.
Then, assumptions about their preference and other factors (like the place of residence or previous education) need to be taken.

Another aspect which has to be considered is the performance of the simulation model.
Even though the code was written with runtime in mind the size of the data set had major impacts on the runtime.

As mentioned above, it would be very interesting to test the model in further research projects with other data sets and areas of application.
Maybe, also extensions (like the optimal stopping) could be incorporated in future, more advanced versions of the simulation model.
